\documentclass[11pt]{article}

%%% Useful packages
\usepackage[utf8]{inputenc}
\usepackage[english]{babel}
\usepackage{amsthm}
\usepackage{amssymb}
\usepackage{mathcomp}
\usepackage{amsmath}
\usepackage{natbib}
\usepackage{array}
\usepackage{wrapfig}
\usepackage{multirow}
\usepackage{tabularx}

\newtheorem{ishaan}{Claim}[section]
\newtheorem{lemma}{Lemma}[section]
\renewcommand\qedsymbol{$\blacksquare$}
%%% Sets page margins, overriding 11pt article defaults
\setlength{\topmargin}{0 in}
\setlength{\headheight}{0 in}
\setlength{\headsep}{0 in}
\setlength{\oddsidemargin}{0in}
\setlength{\textwidth}{6.5 in}
\setlength{\textheight}{9 in}

%%% Useful for layout
\newcommand{\VF}{\vspace*{\fill}}
\newcommand{\HF}{\hspace*{\fill}}

%%% shorthand for sets
\newcommand{\R}{\mathbb{R}}
\newcommand{\Z}{\mathbb{Z}}
\newcommand{\N}{\mathbb{N}}
\newcommand{\Q}{\mathbb{Q}}

\DeclareMathOperator{\Ker}{Ker}

\begin{document}

{\noindent\Large\textbf{Homework 2 due Wed, Sept~8th by 11am in Gradescope}}

\vspace{.25in}

{\large
\noindent
\textbf{Name: Sean Eva} \smallskip \\
\textbf{GTID: 903466156} \smallskip \\
%%% List anyone with whom you discussed the problems
\textbf{Collaborators:} \smallskip \\
%%% List all resources used OTHER than the textbook, lecture notes, 
%%% and previous homework assignments.
\textbf{Outside resources:} \smallskip
}

\pagebreak 

%%% Edit the directions below to remove anything which is not part of
%%% your solution.  A blank line starts a new paragraph.  '\\' is a 
%%% line break.   The default latex formatting is usually acceptable,
%%% so don't spend a significant amount of time making your solutions
%%% look pretty.  Do spellcheck.  PROOFREAD, more than once!

INSERT a ``pagebreak'' command between each problem (integer numbers).
Problem subparts (letter numbered) can be on the same page.

REMOVE all comments (within ``textit\{\}'' commands) before submitting
solutions.

DO NOT include any identifying information (name, GTID) except on the
first/cover page.

\begin{enumerate}

\item Let $G$ be a group with subgroups $A$ and $B$.  Let 
$AB = \{ab \mid a \in A,\ b \in B\}$.
\begin{enumerate}
\item If $G$ is abelian, prove that $AB$ is a subgroup of $G$.

\begin{proof}
Let $G$ be a group such that $G$ is abelian and $A, B$ are subgroups of $G$. Let $AB = \{ ab\mid a\in A, b\in B\}.$ Since $A, B$ are subgroups, $e\in A, e\in B$, so $e=e\circ e\in AB$. Let $a_1b_1, a_2b_2\in AB$, then we know that $(a_1b_1)(a_2b_2)=(a_1a_2)(b_1b_2)\in AB$ since $a_1a_2\in A, b_1b_2\in B$ and $G$ is abelian, which implies that $AB$ is closed under the inherited operation from $G$. If $ab\in AB$, then $(ab)^{-1}=a^{-1}b^{-1}\in AB$ which implies that $AB$ contains inverses. Therefore, $AB$ is a subgroup of G.
\end{proof}

\item Suppose $b^{-1} A b \subseteq A$ for all $b \in B$.  
Show $AB$ is a subgroup of $G$.

\begin{proof}
Let $G$ be a group such that $A, B$ are subgroups of $G$. Let $AB= \{ab\mid a\in A, b\in B\}.$ Since $A, B$ are subgroups, $e\in A, e\in B$, so $e=e\circ e\in AB$. Consider $a\in A, b\in B$ then $(ab)^{-1}=b^{-1}a^{-1}=(b^{-1}a^{-1}b)b^{-1}$ and since $b^{-1}Ab\subseteq A$ when we know that $(b^{-1}a^{-1}b)\in A$ and that $AB$ contains inverses. Additionally, consider $a, a'\in A$ and $b, b'\in B$. Consider $aba'b'=a(ba'b^{-1})bb'$. Since $b^{-1}Ab\subseteq A$ we know that $ba'b^{-1}\in A$ and we know that $a(ba'b^{-1})\in A$ and $bb'\in B$ then we know that $AB$ is closed under the operation and that $AB$ is a subgroup of $G$. 
\end{proof}

\item Disprove that $AB$ is always a subgroup of $G$.

Consider if $G=S_3$, $A=<(12)>$ and $B=<(23)>.$ A and B are subgroups of $G$. However, $AB=\{1, (12), (23), (132)\}$. By Lagrange's Theorem, since $|AB|=4$ and $|G|=6$ and $4$ does not divide $6$ then $AB$ is not a subgroup of $G$.

\end{enumerate}



\pagebreak

\item Let $G$ be a group and $H$ a subgroup of $G$.  
\begin{enumerate}
\item Suppose $a^{-1} H a \subseteq H$ for all $a \in G$.  
Prove $a^{-1} H a = H$.

\begin{proof}
Consider $h\in H$ and $a\in G$. There exists some $h'\in H$ such that $ha=ah'$. Then $a^{-1}ha=h'\in H$. Since this applies to any $h\in H$, $a^{-1}Ha=H$.
\end{proof}

\item Suppose every right coset of $H$ in $G$ is also a left one.  
Prove $aHa^{-1} = H$.

\begin{proof}
Consider $aha^{-1}$ for $h\in H$. We know that $ah\in aH=Ha,$ so therefore $ah=h'a$ for some $h'\in H$. Therefore, $aha^{-1}=h'$. Since this applies to any $h\in H$, then $aHa^{-1}=H$.
\end{proof}

\end{enumerate}
\pagebreak

\item Let $(G, \ast)$ and $(G', \circ)$ be two groups with identity 
$e$ and $e'$ respectively.
Let $\phi: G \rightarrow G'$ be a homomorphism.
\begin{enumerate}
\item Prove that $\phi(G)$, the image of $G$, is a subgroup of $G'$.

\begin{proof}
In order to prove that the image of $G$ is a subgroup of $G'$ we need to prove that the image is closed, contains the identity, and contains inverses. Consider $a,b\in G$. Then $\phi(a), \phi(b)\in \phi(G).$ Therefore, since $\phi$ is a homomorphism, $\phi(ab)=\phi(a)\phi(b)\in\phi(G)$ which implies that the image of $G$ is closed. Let $a\in G.$ Then $a=ae$, but then $\phi (a)=\phi(ae)=\phi(a)\phi(e)$  by the definition of a homomorphism so by left cancellation in $G'$ we have that $e'=\phi(e).$ Also, since $e'=\phi(e)=\phi(aa^{-1})= \phi(a)\phi(a^{-1}).$ Therefore, the inverse of $\phi(a)$ in $G'$ is $\phi(a^{-1})$; $\phi(a)^{-1}=\phi(a^{-1}).$ Therefore, the image of $G$ is a subgroup of $G'.$
\end{proof}

\item Prove that $\phi$ is a monomorphism if and only if $\Ker \phi = (e)$.

\begin{proof}
($\Rightarrow$) Let $\phi$ be a monomorphismm that is to say that for $a,b\in G$ such that $\phi(a)=\phi(b)$ that $a=b$ and since $\phi$ is a homomorphism $\phi(e)=e'$. If $g\in \Ker(f),$ then we have that $\phi(g)=e'$. Therefore, $\phi(g)=\phi(e).$ Since $\phi$ is a monomorphism then $g=e$ and then $\Ker(\phi)=\{e\}.$\\
($\Leftarrow$) Let $\Ker(\phi)=\{e\}$. Let $a,b\in G$ such that $\phi(a)=\phi(b)$. Then we have that
\begin{align*}
    \phi(ab^{-1}) &= \phi(a)\phi(b^{-1})\\
    &= \phi(a)\phi(b)^{-1}\\
    &= \phi(a)\phi(a)^{-1}\\
    &= e'.
\end{align*} Thus, the element $ab^{-1}$ is in the $\Ker(\phi)=\{e\}$ and hence $ab^{-1}=e.$ This implies that $a=b$ and that $\phi$ is a monomorphism.
\end{proof}

\end{enumerate}
\pagebreak

\item Let $G$ be a group and $H$ a subgroup of $G$.  
Let $S$ be the set of all distinct left cosets of $H$ in $G$ and 
$T$ the set of all distinct right cosets of $H$ in $G$.
\begin{enumerate}
\item Prove or disprove: If $aH, bH \in S$ and $aH \neq bH$,
then $Ha \neq Hb$.\\

This statement is false. Consider $H=\{e, (12)\}$ from $S_3$. Then, $(13)H=(123)H=\{(13), (123)\}$. While, $H(13)=\{(13),(132)\}$ and $H(123)=\{(23),(123)\}$. Therefore, $H(123)\neq H(13)$.

\item Prove there is a 1-1 mapping of $S$ onto $T$.

Consider the mapping $x\rightarrow x^{-1}$. Any element of the form $xh$ for $h\in H$ gets mapped to $h^{-1}x^{-1}$ which lies in $Hx^{-1}$. Therefore, the image of $xH$ under the mapping is inside $Hx^{-1}$. Then, any element in $Hx^{-1}$ comes as the image of the unique element under the inverse mapping and that element is in $xH$ if $hx^{-1}$ is an element, it comes as the image of the element $xh^{-1}\in xH$

\end{enumerate}

\end{enumerate}

\end{document}

