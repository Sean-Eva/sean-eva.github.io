\documentclass[11pt]{article}

%%% Useful packages
\usepackage{amsfonts}
\usepackage{amssymb}
\usepackage{amsmath}
\usepackage{amsthm}

%%% Sets page margins, overriding 11pt article defaults
\setlength{\topmargin}{0 in}
\setlength{\headheight}{0 in}
\setlength{\headsep}{0 in}
\setlength{\oddsidemargin}{0in}
\setlength{\textwidth}{6.5 in}
\setlength{\textheight}{9 in}

%%% Useful for layout
\newcommand{\VF}{\vspace*{\fill}}
\newcommand{\HF}{\hspace*{\fill}}

%%% shorthand for sets
\newcommand{\R}{\mathbb{R}}
\newcommand{\Z}{\mathbb{Z}}
\newcommand{\N}{\mathbb{N}}
\newcommand{\Q}{\mathbb{Q}}

\DeclareMathOperator{\Ker}{Ker}

\begin{document}

{\noindent\Large\textbf{Homework 5 due Wed, Oct~20th by 11am in Gradescope}}

\vspace{.25in}

{\large
\noindent
\textbf{Name: Sean Eva} \smallskip \\
\textbf{GTID: 903466156} \smallskip \\
%%% List anyone with whom you discussed the problems
\textbf{Collaborators:} \smallskip \\
%%% List all resources used OTHER than the textbook, lecture notes, 
%%% and previous homework assignments.
\textbf{Outside resources:} \smallskip
}

\pagebreak 

%%% Edit the directions below to remove anything which is not part of
%%% your solution.  A blank line starts a new paragraph.  '\\' is a 
%%% line break.   The default latex formatting is usually acceptable,
%%% so don't spend a significant amount of time making your solutions
%%% look pretty.  Do spellcheck.  PROOFREAD, more than once!

INSERT a ``pagebreak'' command between each problem (integer numbers).
Problem subparts (letter numbered) can be on the same page.

REMOVE all comments (within ``textit\{\}'' commands) before submitting
solutions.

DO NOT include any identifying information (name, GTID) except on the
first/cover page.

\begin{enumerate}

\item Problem 4.1 \# 20.  \emph{Hint: Compute a product table.}

\[ \begin{array}{r|*{8}{r}}
& 1 & i & j & k & -1 & -i & -j & -k \\ \hline
1 & 1 & i & j & k & -1 & -i & -j & -k \\
i & i & -1 & k & -j & -i & 1 & -k & j \\
j & j & -k & -1 & i & -j & k & 1 & -i \\
k & k & j & -i & -1 & -k & -j & i & 1 \\
-1 & -1 & -i & -j & -k & 1 & i & j & k \\
-i & -i & 1 & -k & j & i & -1 & k & -j \\
-j & -j & k & 1 & -i & j & -k & -1 & i \\
-k & -k & -j & i & 1 & k & j & -i & -1 \\
\end{array} \]
\begin{enumerate}
    \item 
    
    Let assumptions be as in problem 4.1 \# 20. In order to show this is a group, we need to show that it is nonempty, closed under the operation, contains an identity, contains inverses, and is associative. The identity of this group is $1$ by the definition of quaternions. Similarly, by the definition of quaternions it is nonempty. By creating the product table above, we can see that the group is closed under multiplication. We can also see that for every element $a$, there is another element $b$ such that $ab = 1$, the identity. Specifically, $-1$ is its own inverse, $i$ is the inverse of $-i$, $j$ is the inverse of $-j$, and $k$ is the inverse of $-k$. It is also then true by the definition of quaternions that for $x, y, z\in G$ that $(xy)z=x(yz)$ and is therefore associative. Since the quaternions are nonempty, contain an identity, contains inverses, is closed under multiplication, and are associative they are a group under the operation of multiplication.
    
    \item
    
    There is the obvious trivial subgroup $<1>$. Then we have the cyclic groups $<-1>, <i>, <j>, <k>, <-i>, <-j>, <-k>$. However, if we write out $<i> = \{1, i, -1, -i\}$ and similarly for $<j>$ and $<k>$. This then means that $<i>=<-i>$. Therefore, the subgroups of $G$ are $<1>,<-1>,<i>,<j>,$ and $<k>.$
    
    \item
    
    The center of $G$ is $\{a\in G|ax = xa, \forall x\in G\}$. By inspecting the product table above, this is true for $1$ and $-1$. Therefore, $Z(G)=\{1, -1\}$.
    
    \item
    
    It is easy to see that $G$ is nonabelian. For example, $i*j = k \neq j * i = -k$. However, the subgroup $<-1> = \{1, -1\}$ we can see that $\forall x\in G, x*1*x^{-1} = x*x^{-1} = 1 \in <-1>,$ and $x*-1*x^{-1} = -x*x^{-1} = -1\in <-1>$ which then shows that $<-1>$ is a normal subgroup. Then we can show that $<i>, <j>,$ and $<k>$ are normal too by looking at the product table above. This then shows that for the group $G$ which is nonabelian, all its subgroups are normal.
    
\end{enumerate}

\pagebreak

\item Prove that a division ring is a domain directly from the definitions.

Let $R$ be a ring such that $R$ is a division ring. That is to say that $R$ has a unit and that for every $a\in R, a\neq 0$ there is a corresponding $a^{-1}\in R$ such that $a\cdot a^{-1}=a^{-1}\cdot a = 1,$ the unit in $R.$ This means that $R$ has a multiplicative identity. Let $a, b\in R$ such that $a\cdot b = 0$ where $a\neq 0$. This then implies, by the definition of a ring, that $a^{-1}$ exists. Then we can say that $a^{-1}(ab) = a^{-1}(0) = 0 = (1)b = (a^{-1}a) b$. This implies that whenever $ab=0$ either $a=0$ or $b=0$ which means that $R$ is an integral domain.

\pagebreak

\item Give an example, in the quaternions, of a noncommuative domain
that is not a division ring.

Consider $\{a+bi+cj+dk|a, b, c, d\in \Z\}$ which is a subset of the quaternions. It is easy to see that this is a noncommutative domain because $ij\neq ji$, and since it does not contain $2^{-1}$ it cannot be a division ring.

\pagebreak

\item Let $R$ be the ring of $2 \times 2$ matrices over the reals.
Prove that $S = \{ \begin{pmatrix} a & b \\ -b & a \end{pmatrix} \mid 
a, b \in \R \}$ 
is a field.

Let assumptions be as above. In order to prove that $S$ is a field, we need to prove that it is a ring, commutative ring, and that is a division ring. First, in order to prove that $S$ is a ring, or more specifically a substring, we need to show that $ab, a+b, a+(-b)\in S$ for $a,b\in S$. Consider $a = \begin{pmatrix} a & b \\ -b & a\end{pmatrix}, b = \begin{pmatrix} c & d \\ -d & c\end{pmatrix}$ for $a,b,c,d\in \R$, then we can see that $a+b = \begin{pmatrix} a & b \\ -b & a\end{pmatrix} + \begin{pmatrix} c & d \\ -d & c\end{pmatrix} = \begin{pmatrix} a + c & b + d \\ -b + -d & a + c \end{pmatrix} = \begin{pmatrix} a + c & b + d \\ -(b + d) & a + c \end{pmatrix}\in S$. Similarly, $a + (-b) = \begin{pmatrix} a & b \\ -b & a\end{pmatrix} + \begin{pmatrix} -c & -d \\ d & -c\end{pmatrix} = \begin{pmatrix} a - c & b - d \\ -b + d & a - c \end{pmatrix} = \begin{pmatrix} a - c & b - d \\ -(b - d) & a - c \end{pmatrix}\in S$. Lastly, $ab = \begin{pmatrix} a & b \\ -b & a\end{pmatrix} \begin{pmatrix} c & d \\ -d & c\end{pmatrix} = \begin{pmatrix} ac - bd & ad + bc \\ -bc - ad & -bd + ac \end{pmatrix} = \begin{pmatrix} ac - bd & ad + bc \\ -(ad + bc) & ac - bd \end{pmatrix}\in S$. Therefore, since for some $a,b\in S$ we found that $ab, a+b, a + (-b)\in S$, we know that $S$ is a subring of $R$. We will next show that $S$ is commutative. For $a, b\in S$, we can see that $ab = \begin{pmatrix} a & b \\ -b & a\end{pmatrix} \begin{pmatrix} c & d \\ -d & c\end{pmatrix} = \begin{pmatrix} ac - bd & ad + bc \\ -bc - ad & -bd + ac \end{pmatrix} = ba\in S$. Thus, $S$ is commutative. Lastly, we need to show that $S$ is a division ring. It would be helpful to first find the unit element of $S$. This is trivial in this ring because it is just $1 = \begin{pmatrix} 1 & 0 \\ 0 & 1 \end{pmatrix}$ the identity matrix. Consider $a = \begin{pmatrix} a & b \\ -b & a\end{pmatrix}$, using the properties of matrices $\det(a) = a^2+b^2 \neq 0$ if either $a$ or $b\neq 0$ if this is true, then $a^{-1}$ exists in the form $\frac{1}{a^2+b^2}\begin{pmatrix} a & -b \\ b & a\end{pmatrix} = \begin{pmatrix} \frac{a}{a^2+b^2} & \frac{-b}{a^2+b^2} \\ \frac{b}{a^2+b^2} & \frac{a}{a^2+b^2}\end{pmatrix}\in S$. By definition of matrices $aa^{-1} = 1.$ If $a, b= 0$, $\begin{pmatrix}0 & 0 \\ 0 & 0 \end{pmatrix}$ which is the $0$ or the additive identity in $S$ and by the definition of a division ring is not included. Therefore, $S$ is a division ring. Since $S$ is a ring, is commutative, and is a division ring, we know that $S$ is a field.

\pagebreak

\item Let $R$ be a finite integral domain.  Prove that $R$ is a field.

Let assumptions be as above. That is to say that in the ring $R$, if $a\cdot b=0$ then either $a=0$ or $b=0$ for $a,b\in R$. In order to prove that $R$ is a field, $R$ needs to be a ring, $R$ needs to be commutative, and $R$ needs to be a division ring. Given the definition of being an integral domain, $R$ is a ring and $R$ is commutative. That leaves to prove that $R$ is a division ring. Let $1\in R$ be the unit element in $R$. Consider the elements of $R=\{a_1, a_2, a_3, ..., a_n\}$ for $n\in \Z$. Let us take arbitrary $a\in R$ and multiply all elements by $a$. Since $R$ is a finite integral domain, it is true to say then that $R = \{aa_1, aa_2, aa_3, ..., aa_n\}$. For some $a_i$ where $1 \leq i \leq n$ we have that $aa_i=1$ which means that $aa^{-1}=1$ has a solution in $R$ and $R$ is a division ring. Therefore, since $R$ is a finite integral domain and we proved that $R$ is a division ring, we know that $R$ is a field.

\pagebreak

\item Problem 4.2 \# 8.

\begin{enumerate}
    \item 
    
    Let assumptions be as in the problem, that is $F$ is a finite field. Let us say that $|F|=n$ and $1\in F$ is the multiplicative identity of $F$. We know that $F=\{1, 2\cdot 1, 3\cdot 1, ..., n\cdot 1\, n+1\cdot 1\}$. However, since $F$ has $n$ elements, we know that there is a non-distinct element in the previous statement such that $k\cdot 1+m\cdot 1=0$ for some $k\neq m$. Then we know that $(k-m)\cdot 1 = 0.$ Let $(k+m)=p$ such that $p$ is the least positive integer such that $p\cdot 1=0$. Now we need to show that $p$ is prime. Suppose not, then $p=ns$ where $1<s<p, 1<n<p$. Then, $p\cdot 1 = p\cdot 1^1 = (ns)\cdot(1\cdot 1) = (n\cdot 1)(s\cdot 1)=0$. Since $F$ is a we know that either $n\cdot 1 = 0$ or $s\cdot 1=0$ which is a contradiction because we said that $p$ is the least integer such that $p\cdot 1=0.$ Therefore, there exists $p$ which is prime such that $p\cdot 1 = 0$. Then we know that for any $a\in F$ we can see that $a+a+a+a...+a=a\cdot 1+a\cdot 1+...+a\cdot1=a(1+1+1+...+1)=a(p\cdot1)=a\cdot0=0.$ Therefore, we know that there exists a prime $p$ such that $pa=0$ for all $a\in F$.
    
    \item
    
    Let assumptions be as in the problem, that is $F$ is a finite field. We proved in part a that there exists a prime $p$ such that $p\cdot 1 = 0.$ Again, let $p$ be the least integer for which $p\cdot 1=0$. Therefore, $1$ has order $p$. Let $p'$ be any prime dividing $|F|=q.$ By Cauchy's Theorem, we know that there is an element $b\in F$ that has order $p'$. Since, from part a, $pa=0$ for all $a\in F$, then we know that $pb=0$. Since the order of $b$ is $p'$, then $p'|p$ which implies that $p'=1$ or $p'=p$ but since $p'$ is prime, we know that $p'\neq 1$, and therefore, $p'=p$. Thus, any prime dividing $q$ must be $p$ and it follows that $q=p^n.$
    
\end{enumerate}

\end{enumerate}

\end{document}

