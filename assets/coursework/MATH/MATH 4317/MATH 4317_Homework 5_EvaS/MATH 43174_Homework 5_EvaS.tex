\documentclass[11pt]{article}

%%%%%%%%%%%%%%%%%%%%%%%%
%%%%%%%%%%%%%%%%%%%%%%%%
%%%%%%Packages
%%%%%%%%%%%%%%%%%%%%%%%%
%%%%%%%%%%%%%%%%%%%%%%%%

\usepackage{amsthm}
\usepackage{amsmath}
\usepackage{amssymb}
\usepackage[margin=1in]{geometry}
\usepackage{enumerate}
%\usepackage{hyperref}
%\usepackage{mathrsfs}
%\usepackage{color}
%\usepackage{bm}



%%%%%%%%%%%%%%%%%%%%%%%%
%%%%%%%%%%%%%%%%%%%%%%%%
%%%%%%amsthm settings
%%%%%%%%%%%%%%%%%%%%%%%%
%%%%%%%%%%%%%%%%%%%%%%%%

\theoremstyle{definition}
\newtheorem{problem}{Problem}
\newtheorem{claim}{Claim}
\newtheorem{definition}{Definition}

%%%%%%%%%%%%%%%%%%%%%%%%
%%%%%%%%%%%%%%%%%%%%%%%%
%%%%%%Custom commands: mathbb
%%%%%%%%%%%%%%%%%%%%%%%%
%%%%%%%%%%%%%%%%%%%%%%%%

\newcommand{\A}{\mathbb A}
\newcommand{\C}{\mathbb{C}}
\newcommand{\D}{\mathbb{D}}
\newcommand{\E}{\mathbb{E}}
\newcommand{\F}{\mathbb{F}}
\newcommand{\N}{\mathbb{N}}
\renewcommand{\P}{\mathbb{P}}
\newcommand{\R}{\mathbb{R}}
\newcommand{\X}{\mathbb{X}}
\newcommand{\Z}{\mathbb{Z}}
\newcommand{\Q}{\mathbb{Q}}

%%%%%%%%%%%%%%%%%%%%%%%%
%%%%%%%%%%%%%%%%%%%%%%%%
%%%%%%Custom commands: greek
%%%%%%%%%%%%%%%%%%%%%%%%
%%%%%%%%%%%%%%%%%%%%%%%%

\renewcommand{\a}{\alpha}
\renewcommand{\b}{\beta}
\newcommand{\g}{\gamma}
\renewcommand{\d}{\delta}
\newcommand{\e}{\epsilon}
\renewcommand{\l}{\lambda}

\begin{document}

\noindent {\bf MATH 4317 Homework 5}  \hfill Due Friday, 3/4/2022 at 11PM to Gradescope

\bigskip

\noindent Homework guidelines: 
\begin{itemize}
\item Each problem I assign, unless otherwise stated, is asking you to prove something. Give a full mathematical proof using only results from class or Wade.
\item Submit a PDF or JPG to gradescope. The grader has $\sim 250$ proofs to grade:  please make his job easier by submitting each problem on a different page. 
\item If you submit your homework in Latex, you get 2\% extra credit. 
\end{itemize}

\subsection*{Problems (5 total, 10 pts each)}

\begin{problem}
Prove or disprove: every function $f : \N \to \R$ is uniformly continuous. 
\end{problem}

Consider the function $f : \N \rightarrow \R$ where $f(x) = x^x$. Consider the point $a\in \N$ and $x=a-1$ then we get that $|x-a| = |a-1-a| = |-1|=1$ which implies that $|f(x)-f(a)|=|f(a-1)-f(a)| =|(a-1)^{a-1}-a^a|$. Now let us consider if $x=a+1$ then we get that, similarly $|x-a|=|a+1-a|=|1|=1$ the same as before, but if we take $|f(x)-f(a)|=|f(a+1)-f(a)|=|(a+1)^{a+1}-a^a|$. It is simple to see that $|(a-1)^{a-1}-a^a| <|(a+1)^{a+1}-a^a|$ which implies that for the same $\delta$ we need different larger $\epsilon$ for exceedingly large $x$. Additionally we find that as we make $a$ larger as well and choose $x$ as we just have the range will become larger and larger as we approach $\infty$. Therefore, we can conclude that $f: \N\rightarrow\R$ where $f(x)=x^x$ is not uniformly continuous and the statement is disproved.

\pagebreak
\begin{problem}
Prove that $f(x) = 1 / (1 + x^2)$ is uniformly continuous on $\R$. 

{\it Hint: You may find it useful to prove that 
\[
\frac{| x| + |y|}{(1 + x^2) ( 1 + y^2)} \leq 2
\]
for all $(x,y) \in \R^2$. 
}
\end{problem}

\begin{proof}
A function $f: E \rightarrow \R$ is uniformly continuous if and only if for every $\epsilon > 0$, there is a $\delta > 0$ such that $|x-a| < \delta$ and that $x, a\in E$ implies that $|f(x) - f(a)| < \epsilon$ and for this problem $E = \R$. Then we we have that
\begin{align*}
    |f(x) - f(a)| &= |\frac{1}{1+x^2}-\frac{1}{1+a^2}|\\
    &= |\frac{a^2 - x^2}{(1+x^2)(1+a^2)}|\\
    &= |x-a|\frac{|x+a|}{(1+x^2)(1+a^2)}\\
    &\leq |x-a|\frac{|x|+|a|}{(1+x^2)(1+a^2)}\\
    &= |x-a|(\frac{|x|}{(1+x^2)(1+a^2)} + \frac{|a|}{(1+x^2)(1+a^2)})\\
    &\leq |x-a|(\frac{1}{2(1+a^2)} + \frac{1}{2(1+x^2)})\\
    &\leq |x-a|.
\end{align*} So if we let $\delta = \epsilon$ from these statements we get that $|f(x)-f(a)| \leq |x-a| < \delta = \epsilon$ as desired.
\end{proof}

\pagebreak
\begin{problem}
Prove that if $a_k \geq 0$ and $\sum_1^\infty a_k$ converges, then 
\[
\sum_{1}^\infty \frac{a_k}{k^p}
\]
converges for all $p \geq 0$. 
\end{problem}

\begin{proof}
Let assumptions be as above. Since we are given that $a_k \geq 0$ and that $\sum_1^\infty a_k$ converges. We can do a comparison test between the $\sum_1^\infty a_k$ and $\sum_1^\infty \frac{a_k}{k^p}$. Since we know that $k\geq 1$ and that $p \geq 0$ implies that $k^p \geq 1$. Further, we then know that $\frac{a_k}{k^p} < a_k$. Therefore, we know that $\sum_1^\infty \frac{a_k}{k^p}$ is less than a convergent series and therefore also converges by the comparison test.
\end{proof}

\pagebreak
\begin{problem}
Let $a_k, b_k \geq 0$ and assume that each of $\sum_1^\infty a_k$ and 
$\sum_1^\infty b_k$ converge. Prove that $\sum_1^\infty a_k b_k$ converges. 
\end{problem}

\begin{proof}
In order to show that this series converges, it suffices to show that it is Cauchy. Since we know that $\sum_1^\infty a_k, \sum_1^\infty b_k$ converge, we know that they are Cauchy. This further implies that the sequence of partial sums of both of these series converge and are also Cauchy. We will represent these partial sums as $s_n, t_n$ where they are the partial sums of $\sum_1^\infty a_k, \sum_1^\infty b_k$. We then know that the product of Cauchy sequences is Cauchy, so we can take $z_n = s_nt_n$, which is then the product of the partial sums of $\sum_1^\infty a_kb_k$, and since this is Cauchy, it fulfills the Cauchy Criterion (Theorem 6.8). Therefore, we know that $\sum_1^\infty$ is Cauchy and further it converges as desired.
\end{proof}

\pagebreak
\begin{problem}
Prove that $\sum_{k = 1}^\infty \frac{(-1)^{k + 1}}{k}$ converges by following these steps: 

\begin{enumerate}
\item First, show that writing $s_n = \sum_1^n (-1)^{k+1} / k$ for the $n$-th partial sums, one has that the sequence $(s_{2n})_{n \geq 1}$ is increasing and bounded from above. 
\item Next, show that the sequence $(s_{2 n + 1})_{n \geq 1}$ is decreasing and bounded from below. 
\item Show that $s_{2n+1} - s_{2n} \to 0$ as $n \to \infty$.
\item Prove, using 1-3, that $(s_n)$ converges. 
\end{enumerate}

\end{problem}

\begin{proof}
Let us consider the partial sums of the series $s_n = \sum_{k=1}^{n}\frac{(-1)^{k+1}}{k}$. We are going to consider specifically the partial sum $(s_{2n})_{n\geq 1} = \sum_{k=1}^{2n}\frac{(-1)^{k+1}}{k}.$ We can notice that $s_{2n} = \sum_{k=1}^{2n}\frac{(-1)^{k+1}}{k} = 1 - \frac{1}{2} + \frac{1}{3} - \frac{1}{4} + ... + \frac{1}{2n-1} - \frac{1}{2n} = (1-\frac{1}{2}) + (\frac{1}{3} - \frac{1}{4}) + ... = \frac{1}{2} + \frac{1}{12} + \frac{1}{30} + ... = \sum_{k = 1}^n (\frac{1}{2k-1}- \frac{1}{2k}) = \sum_{k = 1}^n (\frac{1}{(2k)(2k-1)})$. We will notice a couple things about this series; namely, that it is always increasing with terms that are getting smaller and smaller. Then we can compare this series to the series $\sum_{k = 1}^n \frac{1}{k^2}$ as we will notice that $\sum_{k = 1}^n (\frac{1}{(2k)(2k-1)}) < \sum_{k = 1}^n \frac{1}{k^2}$ which we know converges by the p-series test. By using the comparison test, we then know that $\sum_{k = 1}^n (\frac{1}{(2k)(2k-1)})$ converges and is subsequently bounded from above. Next we could consider the partial sum $(s_{2n+1})_{n\geq 1} = \sum_{k=1}^{2n+1}\frac{(-1)^{k+1}}{k}$. We are going to similarly consider the terms of this series $\sum_{k=1}^{2n+1}\frac{(-1)^{k+1}}{k} = 1 - \frac{1}{2} + \frac{1}{3} - \frac{1}{4} + \frac{1}{5} - ... - \frac{1}{2n} + \frac{1}{2n+1} = 1 - (\frac{1}{2} + \frac{1}{3}) - (\frac{1}{4} + \frac{1}{5}) - .... = 1 - \frac{1}{6} - \frac{1}{20} - ... = 1 - \sum_{k = 1}^n(\frac{1}{2k}-\frac{1}{2k+1}) = 1 - \sum_{k = 1}^n(\frac{1}{(2k)(2k+1)})$; we going to notice a couple things about this partial sum, specifically that it is always decreasing and that the terms being subtracted are getting smaller and smaller with each term. We can compare it with the same series as before so we know that it is convergent and therefore bounded from below. So we can take that $s_{2n+1} - s_{2n} = 1 - \sum_{k = 1}^n(\frac{1}{(2k)(2k+1)}) - \sum_{k = 1}^n (\frac{1}{(2k)(2k-1)}) = 1 - 0 - 1 = 0.$ Therefore we can find that this series is therefore Cauchy and the series converges because it is Cauchy.
\end{proof}

\end{document}
