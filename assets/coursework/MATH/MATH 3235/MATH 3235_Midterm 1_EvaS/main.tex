\documentclass[letterpaper,12pt,addpoints]{exam}
\usepackage{amssymb}
\usepackage{amsmath, nccmath}
\usepackage{epic}
\usepackage{pdfpages}
\usepackage[utf8]{inputenc}
\usepackage[english]{babel}
\usepackage{amsthm}
\usepackage{mathcomp}
\usepackage{natbib}

\newtheorem{ishaan}{Theorem}[section]
\newtheorem{lemma}{Lemma}[section]
\renewcommand\qedsymbol{$\blacksquare$}
\renewcommand{\labelenumii}{\arabic{enumii}.}

\qformat{Question \thequestion\dotfill \emph{\totalpoints\ point}}

\renewcommand{\Pr}{\mathbb{P}}
\begin{document}
 

\header{MATH 3235}{Midterm 1}{Due March 2, 2021 before 2:00pm}
\firstpagefooter{}{}{}
\pagestyle{headandfoot}
\vspace*{1cm}
\begin{center}
\fbox{\parbox{6in}{\centering {\bf This is a take home midterm. You can use 
your notes, my online notes on canvas and the textbooks book. 
You are supposed to work on your own text without external help. I'll be 
available to answer question in person or via email. Please, write clearly and 
legibly and take a readable scan before uploading.}}}
\end{center}



\vspace*{\fill}

\vspace{0.5cm}
\noindent
\makebox[\textwidth]{Name (print): Sean Eva}
\vspace{0.5cm}

\vspace*{\fill}
 
\begin{center}
\gradetable[h][questions]
\end{center}

\vspace{0.5cm}

\vspace*{\fill}
\newpage


\printanswers

\setcounter{page}{1}
\footer{}{Page \thepage\ of \numpages}{}

\begin{questions}
\question


\begin{parts}
\part[10] The number of cars $N$ that arrive at a gas station in an hour is a 
Poisson r.v. with parameter $\lambda$. Each car has a probability $p$ of 
needing service, independently from all other cars. Find the p.m.f of the 
number of car $Q_s$ needing service that arrive in an hour. ({\bf Hint:} Write 
$Q_s$ as a random sum.)


\begin{solution}
\begin{center}
$\sum_{k=0}^{\infty}\frac{1}{k!}\lambda^ke^{-\lambda}$
\end{center}
Therefore the p.m.f for a Poisson r.v. with parameter $\lambda$ in terms of values $N=0, 1, 2, ...$ for the number of cars there are, is
\begin{center}
$\mathbb{P}(N=k)=\frac{1}{k!}\lambda^ke^{-\lambda}$.
\end{center}
\end{solution}

\vspace*{\fill}
\part[10] Let $Q_n$ be the number of cars not needing service that arrive in an 
hour. Show that $Q_s$ and $Q_n$ are independent.

\begin{solution}\\
Since the chance of a car coming in and needing service is independent, that means the chance of a car coming in and not needing service is also independent for each car. Therefore, regardless of the number of cars that come in and need service, the number of cars coming in and do not need service will not be influenced. Therefore, $Q_s$ and $Q_n$ are independent.
\end{solution}


\end{parts}

\vspace*{\fill}\eject

\question


Let  $A,B$ and $C$ be three events. You know that 
\[
B\perp C,\qquad A\perp (B\cup C), \qquad A\perp (B\cap C),\qquad\mathrm{and} 
\qquad 
A\perp (B\cap C^c)
\]
where $B\perp C$ means that $B$ and $C$ are independent.
\begin{parts}

\part[10]
Show that
\begin{align*}
\mathbb{P}(A\cap B\cap C)=&\mathbb{P}(A)\mathbb{P}(B)\mathbb{P}(C)\\
\mathbb{P}(A\cap B\cap C^c)=&\mathbb{P}(A)\mathbb{P}(B)\mathbb{P}(C^c)
\end{align*}
({\bf Hint}: for the second one, show first that, for every $B$ and $C$, if 
$B\perp C$ then $B\perp C^c$)
\begin{solution}\\
First: Showing that $\mathbb{P}(A\cap B\cap C) = \mathbb{P}(A)\mathbb{P}(B)\mathbb{P}(C)$ is the same thing as showing $\mathbb{P}(A\cap (B\cap C))$ since we know that $B\perp C$ (given), and we know that $A\perp (B\cap C)$ (given). We can say that $\mathbb{P}(A\cap (B\cap C)) =\mathbb{P}(A)\mathbb{P}(B\cap C) = \mathbb{P}(A)\mathbb{P}(B)\mathbb{P}(C)$\\
Second: In order to show this, first we would like to show that $B\perp C^c$. To do this, since we know that $B\perp C$
\begin{align*}
    \mathbb{P}(B\cap C^c) &= \mathbb{P}(B\backslash (B\cap C))\\
    &= \mathbb{P}(B)-\mathbb{P}(B\cap C)\\
    &= \mathbb{P}(B)-\mathbb{P}(B)\mathbb{P}(C)\\
    &= \mathbb{P}(B)*(1-\mathbb{P}(C))\\
    &= \mathbb{P}(B)\mathbb{P}(C^c).
\end{align*}
Therefore, $B\perp C^c$. Thus, this is now similar to the first part of this problem. Then, given that $A\perp (B\cap C^c)$
\begin{align*}
    \mathbb{P}(A\cap B\cap C^c) &= \mathbb{P}(A\cap (B\cap C^c))\\
    &=\mathbb{P}(A)\mathbb{P}(B\cap C^c)\\
    &=\mathbb{P}(A)\mathbb{P}(B)\mathbb{P}(C^c).    
\end{align*}
\end{solution}
\vspace*{\fill}

\part[10]
Show that
\[
\Pr(A\cap B)=\Pr(A)\Pr(B)
\]
You can use that $B=(B\cap C)\cup(B\cap C^c)$.

\begin{solution}\\
Given that $B=(B\cap C)\cup(B\cap C^c)$, then,
\begin{align*}
    \mathbb{P}(A\cap B) &= \mathbb{P}(A\cap ((B\cap C)\cup(B\cap C^c)))\\
    &= \mathbb{P}(A\cap (B\cap C) \cup A\cap (B\cap C^c))\\
    &= \mathbb{P}(A\cap B\cap C) + \mathbb{P}(A\cap B\cap C^c)\\
    &= \mathbb{P}(A)\mathbb{P}(B)\mathbb{P}(C) + \mathbb{P}(A)\mathbb{P}(B)\mathbb{P}(C^c)\\
    &= \mathbb{P}(A)\mathbb{P}(B)(\mathbb{P}(C)+\mathbb{P}(C^c))\\
    &= \mathbb{P}(A)\mathbb{P}(B)(\mathbb{P}(C) + (1-\mathbb{P}(C)))\\
    &=\mathbb{P}(A)\mathbb{P}(B)(1)\\
    &= \mathbb{P}(A)\mathbb{P}(B).
\end{align*}
\end{solution}
\vspace*{\fill}\eject

\part[10]
Show that $A$, $B$, and $C$ are independent. ({\bf Hint:} you just miss one 
independence. Use the hint to point a) to reduce the question to a situation 
similar to point b).)
\begin{solution}\\
It was given to us that $B\perp C$ and in part b we solved that $A\perp B$, which therefore means that we only need to solve that $A\perp C$. This would be the same thing as saying that $\mathbb{P}(A\cap C) = \mathbb{P}(A)\mathbb{P}(C)$. First I will show that $C\perp B^c$,
\begin{align*}
    \mathbb{P}(C\cap B^c) &= \mathbb{P}(C\backslash (C\cap B))\\
    &= \mathbb{P}(C)-\mathbb{P}(C\cap B)\\
    &= \mathbb{P}(C)-\mathbb{P}(C)\mathbb{P}(B)\\
    &= \mathbb{P}(C)*(1-\mathbb{P}(B))\\
    &= \mathbb{P}(C)\mathbb{P}(B^c).
\end{align*}
This allows for $C=(C\cap B)\cup(C\cap B^c)$. Then, 
\begin{align*}
    \mathbb{P}(A\cap C) &= \mathbb{P}(A\cap ((C\cap B)\cup(C\cap B^c)))\\
    &= \mathbb{P}(A\cap (C\cap B) \cup A\cap (C\cap B^c))\\
    &= \mathbb{P}(A\cap C\cap B) + \mathbb{P}(A\cap C\cap B^c)\\
    &= \mathbb{P}(A)\mathbb{P}(C)\mathbb{P}(B) + \mathbb{P}(A)\mathbb{P}(C)\mathbb{P}(B^c)\\
    &= \mathbb{P}(A)\mathbb{P}(C)(\mathbb{P}(B)+\mathbb{P}(B^c))\\
    &= \mathbb{P}(A)\mathbb{P}(C)(\mathbb{P}(B) + (1-\mathbb{P}(B)))\\
    &=\mathbb{P}(A)\mathbb{P}(C)(1)\\
    &= \mathbb{P}(A)\mathbb{P}(C).
\end{align*}
Therefore, $A\perp C$ which then means that $A$, $B$, and $C$ are all independent.
\end{solution}
\vspace*{\fill}\eject
\end{parts}



\question[10]

You and a friend of yours both toss a fair coin twice. Let $X_1$ be the number 
of heads you get and $X_2$ be the number of heads your friend gets.
% \begin{parts}
% \part [10]
Compute the p.m.f. of $Y=X_1+X_2$ and $Z=X_1-X_2$.
\begin{solution}\\
Given the outcomes of $X_1$ and $X_2$, $Y$ can take on the values $0, 1, 2, 3, 4$ where there is one way each to get $Y=0$ or $Y=4$, four ways each to get $Y=1$ or $Y=3$, and six ways to get $Y=2$. Therefore,
\begin{center}
    $p_Y(0)=p_Y(4)=\frac{1}{16}$\hspace{1cm} $p_Y(1)=p_Y(3)=\frac{1}{4}$\hspace{1cm} $p_Y(2)=\frac{3}{8}$.
\end{center}
Whereas for $Z$, it can take on the values $-2, -1, 0, 1, 2$ where there are one way each to get $Z=-2$ and $Z=2$, four ways each to get $Z=-1$ and $Z=1$, and six ways to get $Z=0$. Therefore,
\begin{center}
    $p_Z(-2)=p_Z(2)=\frac{1}{16}$\hspace{1cm}$p_Z(-1)=p_z(1)=\frac{1}{4}$\hspace{1cm}$p_Z(0)=\frac{3}{8}$.
\end{center}
\end{solution}
\vspace*{\fill}\eject

\question

You roll a fair die till you get a 6. Let $X$ be the total number of roll and 
$Y$ be the number of 1 you see in this $X$ rolls. 

\begin{parts}
 
\part[10]
Find $\Pr(Y=y|X=x)$ and $\mathbb{E}(Y|X=x)$.

\begin{solution}\\
The number of $1$s rolled follows a binomial distribution, then $\mathbb{P}(Y=y)=x_{c_y}(\frac{1}{6})^y(\frac{5}{6})^{x-y} \Rightarrow \mathbb{P}(Y=y|X=x) = x_{c_y}(\frac{1}{6})^y(\frac{5}{6})^{x-y}$.\\
Since $\mathbb{P}(Y|X=x)$ follows a binomial distribution (given that Y follows a binomial distribution), then $E(Y|X=x) = \frac{1}{6}x$
\end{solution}
\vspace*{\fill}


\part[10] 
Find $\mathbb{E}(Y)$.

\begin{solution}\\
The number $X$ of the total number of rolls will follow a geometric distribution with p.m.f $p(X-x)=q^{y-1}p$ where $p=\frac{1}{6}$ and $q = 1-\frac{1}{6} = \frac{5}{6}$. Therefore, $E(X) = \frac{\frac{5}{6}}{\frac{1}{6}} = 5$
\begin{align*}
    E(Y) &= E(E(Y|X=x))\\
    &= E(\frac{1}{6}X)\\
    &= \frac{1}{6} E(X)\\
    &= \frac{1}{6}*5\\
    &= \frac{5}{6}
\end{align*}
\end{solution}
\end{parts}
\vspace*{\fill}
\eject


\question[10]
A large hall in your building require 100 light bulbs. You receive a shipment 
of $100$ light bulbs and put them on. Each bulb breaks down in the first month 
of operation with probability $p=0.1$. If it survives the first month, it 
breaks down in the second month of operation with probability $p=0.15$. Suppose 
that after two months the number of working bulbs is 80. Obtain the p.m.f. of 
the number $X$ of bulbs that were working after one month.


\begin{solution}\\
Let us define $X$ as being the number of bulbs working after the first month, and $Y$ be the number of bulbs working in the first month (before the last day of the month). It is easy to see that $Y$ can be defined as a binomial p.m.f where $n=100$, $p=1-0.1=0.9$, and $q=1-p$. Then, $X|Y=y$ is a binomial p.m.f where $n=y$ and $p_x=1-0.15=0.85$ and $q_x=1-p_x$,
The p.m.f of X would then look like,
\begin{align*}
    \mathbb{P}(X=x) &= \sum_{x=y}^{n}\mathbb{P}(Y=y)\mathbb{P}(X=x|Y=y)\\
    &=\sum_{y=x}^{n}\binom{n}{y}p^yq^{n-y}*\binom{y}{x}p_x^xq_x^{y-x}\\
    &=\sum_{x=y}^{n}\frac{n!}{y!(c-y)!}*\frac{y!}{x!(y-x)!}*p^yp_x^xq^{n-y}q_x^{y-x}\\
    &=\frac{n!}{x!}\sum_{z=0}^{n-x}\frac{1}{(z)!(n-x-z)!}p^{x+z}p_x^xq^{n-x-z}q_x^z \hspace{1cm} \text{for } y-x=z\\
    &= \frac{n!}{x!(n-x)!}p^xp_x^x\sum_{z=0}^{n-x}\frac{(n-x)!}{z!(n-x-z)!}p^zq_x^zq^{n-x-z}\\
    &= \frac{n!}{x!(n-x)!}p^xp_x^x\sum_{z=0}^{n-x}\binom{n-x}{z}(pq_x)^zq^{n-x-z}\\
    &= \binom{n}{x}(pp_x)^x(q+pq_x)^{n-x}, x=0, 1, ..., n.
\end{align*}
In this solution, $pp_x = 0.9*0.85=0.765$, $q+pq_x=1-pp_x=0.235$, $n=100$. Therefore, the p.m.f of $X$ is a binomial p.m.f with parameters $n=100$ and $p=0.765$.
\end{solution}

\vspace*{\fill}
\eject

\question[10]
Look at the following functions and decides whether they can be the c.d.f. of a 
r.v. In case they are, discuss whether it is a continuous or discrete r.v. and, 
when possible, write the corresponding p.m.f or p.d.f..

\begin{fleqn}[\parindent]
\begin{align*}
 F_1(x)=&\begin{cases}
           0 & x<0\crcr
          \frac13 & 0 \leq x < 1\crcr
          \frac23 & 1 \leq x < 2.5\crcr
          1 & x  \geq 2.5\crcr
        \end{cases}\\ \\
 F_2(x)=&\begin{cases}
           0 & x<-1\crcr
          \frac12(x+1)^2 & -1 \leq x < 0\crcr
          1-\frac12(x-1)^2 & 0 \leq x < 1\crcr
          1 & x  \geq 1\crcr
        \end{cases}\\ \\
 F_3(x)=&\begin{cases}
           0 & x<-1\crcr
           \frac12+\frac13x  & -1 \leq x < 1\crcr
           1 & x  \geq 1\crcr
        \end{cases}\\  \\
 F_4(x)=&\begin{cases}
           0 & x\leq-1\crcr
           \frac12  & -1 < x < 1\crcr
           1 & x  \geq 1\crcr
        \end{cases}      
\end{align*}
\end{fleqn}

\begin{solution}\\
$F_1(x)$: Yes, this can be a c.d.f. Since the c.d.f. is noncontinuous, then it represents a discrete random variable. The corresponding p.m.f is simply the jumps in values. That is $f_1(x)=
\begin{cases}
0 & x<0 \crcr
\frac{1}{3} & x=0\crcr
\frac{1}{3} & x=1 \crcr
\frac{1}{3} & x=2.5
\end{cases}$\\
$F_2(x)$: Yes, this can be a c.d.f. Since the c.d.f is continuous, then it represents a continuous random variable whose p.d.f is the derivative of $F_2(x)$. Which is, $f_2 = 
\begin{cases}
0 & x < -1\crcr
x+1 & -1\leq x < 0\crcr
1-x & 0\leq x < 1\crcr
0 & x\geq 1
\end{cases}$
\\
$F_3(x)$: This is not a c.d.f because the proposed c.d.f is non-continuous but is also not a simple step function, it cannot be a c.d.f for a r.v.\\
$F_4(x)$: This is not a c.d.f because the middle portion of the proposed c.d.f does not have a closed end and therefore cannot be a c.d.f for a r.v.
\end{solution}


\end{questions}
\vspace*{\fill}
\eject

\end{document}