\documentclass{article}
\usepackage[utf8]{inputenc}
\usepackage[english]{babel}
\usepackage{amsthm}
\usepackage{amssymb}
\usepackage{mathcomp}
\usepackage{amsmath}
\usepackage{natbib}
\usepackage{array}
\usepackage{wrapfig}
\usepackage{multirow}
\usepackage{tabularx}
\usepackage{multirow}
\usepackage{graphicx}

\newtheorem{ishaan}{Theorem}[section]
\newtheorem{lemma}{Lemma}[section]
\renewcommand\qedsymbol{$\blacksquare$}

\title{Homework 12}
\author{Sean Eva}
\date{October 2021}

\begin{document}

\maketitle

\begin{enumerate}
    \item 
    
    Number of hydrogen in sun: $\frac{1.989*10^{30}}{1.674*10^{-27}}=1.188*10^{57}$.\\ Energy released $13*1.188*10^{57} = 1.544*10^{58}\text{eV} = \frac{1.544*10^{58}}{1.602*10^{12}} = 9.640*10^{45}\text{erg}$.\\ Solar luminosity: $3.826*10^{33} \frac{\text{erg}}{\text{s}}$. Then, $\frac{9.640*10^{45}}{3.826*10^{33}} = 2.520*10^{12}\text{s} = \frac{2.520*10^{12}}{60*60*24*365.25} = 7.985*10^4\text{ years}$.\\
    This is much less than the age of the Earth. This means that the Sun's energy is not entirely chemical, especially in this scenario. This energy would only sustain the sun for a fraction of the amount of time that it has been alive.
    \item
    
    \begin{enumerate}
        \item 
        
        $\Delta m = (12.00000+12.00000)-(23.98504) = 0.01496$.\\
        $E = mc^2 = \frac{0.01496*(1.67*10^{-27})*(3*10^8)^2}{1.6*10^{-13}} = 14.053 \text{ MeV}$\\
        Since the initial mass is greater than the final mass, the reaction is exothermic.
        
        \item
        
        $\Delta m = (12.00000+12.000000)-(15.99491+2(4.002603)) = -0.000116$\\
        $E = mc^2 = \frac{-0.000116*(1.67*10^{-27})*(3*10^8)^2}{1.6*10^{-13}} = -0.109 \text{ MeV}$\\
        Since the initial mass is less than the final mass, the reaction is endothermic.
        
        \item
        
        $\Delta m = (18.99840 + 1.007825) - (15.99491 + 4.002603) = 0.008712$\\
        $E = mc^2 = \frac{0.008712*(1.67*10^{-27})*(3*10^8)^2}{1.6*10^{-13}} = 8.184 \text{ MeV}$\\
        Since the initial mass is greater than the final mass, the reaction is exothermic.
        
    \end{enumerate}
    
    \item
    
    \begin{enumerate}
        \item 
        
        $27 \| v$
        
        \item
        
        $27 \| 2 \| \text{He}$
        
        \item
        
        $\gamma$
        
    \end{enumerate}
    
    \item
    
    Total number of Copper 63 atoms: $\frac{3.2}{62.92960*(1.6605402*10^{-24}} = 3.062*10^{22}$ atoms.\\
    $\Delta m = ((29(1.00783)+34(1.00867))-62.92960)*3.062*10^{22} = 1.813*10^{22}$\\
    $E=mc^2 = \frac{(1.813*10^{22})*(1.67*10^{-27})*(3*10^8)^2}{1.6*10^{-13}} = 1.703*10^{25} \text{ MeV}$
    
    \item
    
    $\frac{10.0*3.27}{2*(2.014102)(1.661*10^{-27})} = 4.887*10^{27}\text{ MeV} = 4.887*10^{27}*1.602*10^{-13} = 7.829*10^{14}\text{J}$\\
    $t = \frac{7.829*10^{14}}{100} = 7.829*10^{12}\text{s} = \frac{7.829*10^{12}}{60*60*24*365.25} = 2.5*10^5$ years.
    
    \item
    
    $E = mc^2 = \frac{(42.958770-41.958630-1.00866)*(1.67*10^{-27})*(3*10^8)^2}{1.6*10^{-13}} = -8.003475\text{ MeV}$.\\
    The amount of energy required to remove a neutron from Ca43 is around $8.003475$ MeV.
    
    \item
    
    $E=mc^2 = \frac{(9.012182+4.002603-12.00000-1.008665)*(1.67*10^{-27})*(3*10^8)^2}{1.6*10^{-13}} = 5.748975\text{ MeV}$
    
    \item
    
    \begin{enumerate}
        \item 
        
        $E=mc^2 = \frac{(1(1.00727647)+1(1.008665)-2.014102)*(1.67*10^{-27})*(3*10^8)^2}{1.6*10^{-13}} = 1.728 \text{ MeV}$ or $0.864\text{ MeV}/$nucleon.
        
        \item
        
        $E=mc^2 = \frac{(2(1.00727647)+2(1.008665)-4.002603)*(1.67*10^{-27})*(3*10^8)^2}{1.6*10^{-13}} = 27.505\text{ MeV}$ or $6.876\text{ MeV}/$nucleon
        
        \item
        
        $E=mc^2 = \frac{(26(1.00727647)+30(1.008665)-55.934939)*(1.67*10^{-27})*(3*10^8)^2}{1.6*10^{-13}} = 483.026\text{ MeV}$ or $8.625\text{ MeV}/$nucleon
        
        \item
        
        $E=mc^2 = \frac{(92(1.00727647)+146(1.008665)-238.050786)*(1.67*10^{-27})*(3*10^8)^2}{1.6*10^{-13}} = 1769.538\text{ MeV}$ or $7.435\text{ MeV}/$nucleon.
        
    \end{enumerate}
    
    \item
    
    Part 1: $E=mc^2 = \frac{(2(1.007825)-2.014102-0.0005486)*(1.67*10^{-27})*(3*10^8)^2}{1.6*10^{-13}} = 0.939\text{ MeV}$\\
    Part 2: $E=mc^2 = \frac{(2(0.0005486)-0)*(1.67*10^{-27})*(3*10^8)^2}{1.6*10^{-13}} = 1.031 \text{ MeV}$\\
    Part 3: $E=mc^2 = \frac{(2.014102+1.007825-3.016029)*(1.67*10^{-27})*(3*10^8)^2}{1.6*10^{-13}} = 5.540 \text{ MeV}$\\
    Part 4: $E=mc^2 = \frac{(2(3.016029)-4.002603-2(1.007825))*(1.67*10^{-27})*(3*10^8)^2}{1.6*10^{-13}} = 12.968\text{ MeV}$\\
    Total: $0.939+1.031+5.540+12.968 = 20.478\text{ MeV}$
    
\end{enumerate}

\end{document}
