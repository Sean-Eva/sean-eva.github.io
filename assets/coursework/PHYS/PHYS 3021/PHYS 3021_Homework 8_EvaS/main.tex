\documentclass{article}
\usepackage[utf8]{inputenc}
\usepackage[english]{babel}
\usepackage{amsthm}
\usepackage{amssymb}
\usepackage{mathcomp}
\usepackage{amsmath}
\usepackage{natbib}
\usepackage{array}
\usepackage{wrapfig}
\usepackage{multirow}
\usepackage{tabularx}
\usepackage{multirow}
\usepackage{graphicx}

\newtheorem{ishaan}{Theorem}[section]
\newtheorem{lemma}{Lemma}[section]
\renewcommand\qedsymbol{$\blacksquare$}

\title{Homework 9}
\author{Sean Eva}
\date{September 2021}

\begin{document}

\maketitle

\begin{enumerate}
    \item 
    
    Absolute Magnitude: $m-M = 5\log(\frac{d}{10})$. According to the colorful diagram, an M0 Ib star has absolute magnitude of about $-4$. Therefore,
    \begin{align*}
        m-M &= 5\log(\frac{d}{10})\\
        1-(-4) &= 5\log(\frac{d}{10})\\
        d &= 100\text{pc}
    \end{align*}
    
    \item
    
    Stefan-Boltzmann Law: $L\propto R^2T^4$. Sun temp: $6000k$, Radius $R_\odot$, Mass $M_\odot$. Then for an M0 V star
    \begin{align*}
        L &= (\frac{R_{M0V}}{R_\odot})^2(\frac{T}{6000k})^4\\
        10^{-2} &= (\frac{R_{M0}}{1})^2(\frac{4400}{6000})^4\\
        R_{M0V} &= 0.19 R_\odot.
    \end{align*} And for an M0 Ia star,
    \begin{align*}
        L &= (\frac{R_{M0Ia}}{R_\odot})^2(\frac{T}{6000k})^4\\
        10^5 &= (\frac{R_{M0Ia}}{1})^2(\frac{4400}{6000k})^4\\
        R_{M0Ia} &= 588.03 R_\odot.
    \end{align*}
    
    \item
    
    Stefan-Boltzmann Law: $L\propto R^2T^4$. Betelgeus(4000k, M0 Ib), luminosity: $10^{3.5}$
    \begin{enumerate}
        \item 
        
        \begin{align*}
            \frac{10^{3.5}}{10^2} &= (R)^2(\frac{4000}{4000})^4\\
            R &= 5.62 \text{ times larger}
        \end{align*}
        
        \item
        
        \begin{align*}
            \frac{10^{3.5}}{10^{4.2}} &= (R)^2(\frac{4000}{25000})^4\\
            R &= 17.448 \text{ times larger}
        \end{align*}
        
        \item
        
        \begin{align*}
            \frac{10^{3.5}}{1} &= (R)^2(\frac{4000}{5800})^4\\
            R &= 118.23\text{ times larger}
        \end{align*}
        
    \end{enumerate}
    
\end{enumerate}

\end{document}
